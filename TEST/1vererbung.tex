\newpage
\section{Vererbung}

\begin{enumerate}
    \item Benennen Sie (a) und (b): Java erlaubt für (a) nur die 
        einfache Vererbung. Für (b) gilt diese Einschränkung nicht.
    \item Erläutern Sie, wozu man abstrakte Klasen verwendet.
        Nennen Sie was man bei der Instantiierung eines Objektes einer 
        abstrakten Klasse beachten muss.
    \item Erstellen Sie eine \verb?public? abstrakte Methode mit dem 
        Namen \verb?hello? welche einen \verb?String? zurückgibt und
        keine Parameter hat.
        Was bewirkt diese abstrakte Mehtode für die Klasse?
    \item Erstellen Sie ein \verb?Interface? mit dem Namen 
        \verb?MyInterface?. Dieses hat  nur eine Methode \verb?hello? 
        welche einen \verb?String? zurückgibt und keine Parameter annimmt.
        Was können Sie über die Sichtbarkeit der Methode \verb?hello()?
        sagen, wenn keine Zugriffsmodifikatoren (\verb?public?, 
        \verb?private?, ...) angegeben werden?
    \item OFFENE FRAGE
    \item Nennen Sie je zwei Unterschiede zwischen
        \begin{enumerate}[label=(\alph*)]
            \item Konkreten Klassen und abstrakten Klassen
            \item Abstrakten Klassen und Interfaces
        \end{enumerate}
    \item Zeigen Sie wie die folgenden Vererbungen zu implementieren sind.
        \begin{enumerate}[label=(\alph*)]
            \item Das \verb?Interface? \verb?D? erbt von \verb?A, B, C?.
                Notieren Sie die relevante Code-Zeile. Sind \verb?A, B, C?
                Klassen oder Interfaces?
            \item Die Klasse \verb?X? soll von \verb?Y? erben und \verb?Z?
                implmentieren. Notieren Sie die relevante Code-Zeile.
                Sind \verb?Y, Z? jeweils Klassen oder Interfaces?
            \item Erstellen Sie zu den Teilaufgaben (a) und (b) je ein 
                entsprechendes Klassendiagramm.
        \end{enumerate}
    \item Die Methode \verb?hello()? der Klasse \verb?A?, welche 
        \verb?public? ist und einen \verb?String? zurückgibt, soll 
        \begin{enumerate}[label=(\alph*)] 
            \item in der Klasse \verb?B? überschrieben werden. Die 
                Implementierungen können Sie jeweils selber definieren.
            \item überladen werden mit einer Methode die einen Parameter
                vom Typ \verb?String? annimmt und einer Methode welche
                zwei Parameter vom Typ \verb?String? annimmt.
        \end{enumerate}
    \item Die Klasse \verb?Enzo? erbt von \verb?Ferrari? und diese 
         erbt von \verb?Auto?.
        \begin{enumerate}[label=(\alph*)]
            \item Substituieren Sie ein Auto mit einem Ferrari.
            \item Eine Klasse \verb?Fiat? erbt von \verb?Auto? und es 
                ist die folgende Zeile gegeben: \\\\
                \verb?Auto a = new Ferrari();?\\\\
                Weisen Sie nun einem \verb?Fiat b? das \verb?Auto a? zu 
                mittels eines Casts. Was gilt es hier zu beachten bezüglich 
                der Kompilier- bzw. Laufzeit?
        \end{enumerate}
    \item NOPE
    \item 
        \begin{enumerate}[label=(\alph*)]
            \item Erklären Sie was die Methode \verb?toString()? macht 
                und welche Klasse diese definiert.
            \item Erstellen Sie eine Klasse \verb?A? welche die Methode
                \verb?toString()? so verändert, dass diese 
                \verb?"Java ist auch nur eine Insel"? zurückgibt.
        \end{enumerate}
\end{enumerate}
