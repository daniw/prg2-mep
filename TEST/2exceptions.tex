\newpage
\section{Exceptions}

\begin{enumerate}
    \item Eine Klasse \verb?A? soll eine \verb?IOException? werfen können. 
        \begin{enumerate}[label=(\alph*)]
            \item Erstellen Sie die relevante Zeile der Kalsse \verb?A? die
                dies definiert.
            \item Eine Methode in der Klasse \verb?A? soll diese Exception
                werfen können. Erstellen Sie die relevante Zeile.
        \end{enumerate}
    \item Nennen Sie drei Vorteile des Exception Handling.
    \item Nennen Sie fünf Exceptions (Klassennamen).
    \item Erstellen Sie die relevante Zeile für die Erstellung einer eigenen
        Excpetion mit den Namen MyException welche
        \begin{enumerate}[label=(\alph*)]
            \item unchecked ist
            \item checked ist
        \end{enumerate}
    \item Zeichnen Sie ein Klassendiagramm mit folgenden Elementen
        \begin{itemize}
            \item \verb?Throwable?
            \item \verb?Exception?
            \item \verb?Error?
            \item \verb?RuntimeException?
            \item \verb?MyCheckedException? (eigens erstellte Exception, checked)
            \item \verb?MyUncheckedException? (eigens erstellte Exception, unchecked)
        \end{itemize}
    \item Beschreiben Sie in kurzen Sätzen die Funktion und Bedeutung
        der folgenden Keywords.
        \begin{enumerate}[label=(\alph*)]
            \item \verb?throws?
            \item \verb?throw?
            \item \verb?try?
            \item \verb?catch?
            \item \verb?finally?
        \end{enumerate}
    \item Erläutern Sie den Unterschied zwischen checked und unchecked 
        Exceptions und erklären Sie wie man bestimmt, ob eine eigens
        erstellte Exception checked oder unchecked ist.
\end{enumerate}
