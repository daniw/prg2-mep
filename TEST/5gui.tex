\newpage
\section{GUI Programmierung}

\begin{enumerate}
    \item Nennen Sie drei Kategorien von Elemente, aus welchen eine
        graphische Benutzeroberfläche zusammengestellt wird.
    \item Erklären Sie die Methoden \verb?paint()? und \verb?repaint()?
        anhand der folgenden Fragen.
        \begin{enumerate}[label=(\alph*)]
            \item Was definiert die Methode \verb?paint()??
            \item Welche Parameter hat die Methode \verb?paint()??
            \item Was ruft die Methode \verb?paint()? auf und wann 
                geschieht dies?
            \item Wozu dient die Methode \verb?repaint()??
        \end{enumerate}
    \item Beschreiben Sie den allgemeinen Ablauf eines 
        \begin{enumerate}[label=(\alph*)]
            \item sequentiellen Programms
            \item ereignisgesteuerten Programms
        \end{enumerate}
    \item Benennen Sie (a) bis (c): Eine Event-(a) definiert ein (b).
        Klassen welche sich für Events dieser (a) interessieren, 
        implementieren dieses (b) und registrieren sich bei der (a).
        Tritt ein Event auf die (a) ein, so werden die Methoden bei
        den Listenern der (a) ausgeführt.
    \item Nennen Sie je zwei
        \begin{enumerate}[label=(\alph*)]
            \item Elementare GUI-Komponenten
            \item Container
            \item Layout-Manager
        \end{enumerate}
    \item Nennen Sie zu Swing je zwei 
        \begin{enumerate}[label=(\alph*)]
            \item Vorteile
            \item Nachteile
        \end{enumerate}
    \item Nennen Sie fünf Swing-Komponenten und erklären Sie in
        Stichworten, wozu diese dienen bzw. was diese darstellen.
    \item NOPE
    \item Erklären Sie was 
        \begin{enumerate}[label=(\alph*)]
            \item innere Klassen sind. 
                Machen Sie ein Beispiel reduziert auf die wesentlichen
                Code-Zeilen.
            \item anonyme Klassen sind. Erläutern Sie wozu diese 
                eingesetzt werden. 
        \end{enumerate}
\end{enumerate}
