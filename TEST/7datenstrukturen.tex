\newpage
\section{Datenstrukturen}

\begin{enumerate}
	\item NOPE
	\item Nennen Sie in drei Punkten den Nutzen des Java Collection 
		Framework
	\item Nennen Sie die zwei Basisinterfaces des Java Collection
        Framework
	\item Nennen Sie fünf konkrete Datenstruktur-Klassen die das
        Java Collection Framework zur Verfügung stellt. 
	\item NOPE
	\item NOPE
	\item NOPE
	\item Nennen Sie zwei herausragende Vorteile von generischen 
        Klassen bei Datenstrukturen
	\item Erstellen Sie für die Nodes einer einfachen LinkedList
        eine generische Klasse. Diese hat zwei Instanzvariablen
        \verb?next? und \verb?data? mit je einer Getter und Setter
        Methode dazu. 
	\item Implementieren Sie eine einfache LinkedList mit den 
        Nodes aus Aufgabe 9. Die LinkedList soll die folgenden
        Methoden aufweisen.\\\\
        \verb?insert()?\\
        \verb?remove()?\\
        \verb?isFull()?\\
        \verb?isEmpty()?
	\item Stellen Sie in einer Skizze dar, wie in einer doppelt
        verketteten List ein Element
        \begin{enumerate}[label=(\alph*)]
            \item hinzugefügt wird
            \item entfernt wird
        \end{enumerate}
	\item Nennen Sie Situationen bei denen der Einsatz einer doppelt
        verketteten Liste besonders zu empfehlen ist.
	\item Implemntieren Sie einen einfachen Stack mit den vier Methoden
        \verb?push()?, \verb?pop()?, \verb?isEmpty()? und \verb?isFull()?.
        Implementieren Sie diesen Stack je einmal mit
        \begin{enumerate}[label=(\alph*)]
            \item einem Array
            \item einer ArrayList
        \end{enumerate}
	\item Implementieren Sie eine einfache Queue mit einer LinkedList 
        (aus dem Java Collections Framework). Die Queue muss dabei die
        Methoden \verb?enqueue()?, \verb?dequeue()?, \verb?isEmpty()? 
        und \verb?ifFull()? bieten.
	\item Nennen Sie je zwei Anwendungen/situationen für den Einsatz
        der Datenstruktur
        \begin{enumerate}[label=(\alph*)]
            \item Stack
            \item Queue
        \end{enumerate}
	\item Skizzieren Sie einen binären Baum welcher die \verb?inorder? 
        Ausgabe\\
        \verb?P-R-O-G-R-A-M-M?\\
        liefert und geben Sie dessen Höhe an.
    \item NOPE
	\item NOPE
	\item NOPE
	\item Skizzieren Sie die Klassenstruktur des Java Collection 
        Framework mit den folgenen Elementen.
        \begin{itemize}
            \item \verb?Map?
            \item \verb?sorted Map?
            \item \verb?Set?
            \item \verb?sorted Set?
            \item \verb?Collection?
            \item \verb?List?
            \item \verb?Queue?
        \end{itemize}
	\item Nennen Sie die drei CollectionView Methoden für Maps und
        beschreiben Sie in Stichworten, was diese machen.
	\item NOPE
\end{enumerate}
